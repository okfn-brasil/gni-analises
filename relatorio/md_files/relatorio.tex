% Options for packages loaded elsewhere
\PassOptionsToPackage{unicode}{hyperref}
\PassOptionsToPackage{hyphens}{url}
%
\documentclass[
]{article}
\usepackage{lmodern}
\usepackage{amsmath}
\usepackage{ifxetex,ifluatex}
\ifnum 0\ifxetex 1\fi\ifluatex 1\fi=0 % if pdftex
  \usepackage[T1]{fontenc}
  \usepackage[utf8]{inputenc}
  \usepackage{textcomp} % provide euro and other symbols
  \usepackage{amssymb}
\else % if luatex or xetex
  \usepackage{unicode-math}
  \defaultfontfeatures{Scale=MatchLowercase}
  \defaultfontfeatures[\rmfamily]{Ligatures=TeX,Scale=1}
\fi
% Use upquote if available, for straight quotes in verbatim environments
\IfFileExists{upquote.sty}{\usepackage{upquote}}{}
\IfFileExists{microtype.sty}{% use microtype if available
  \usepackage[]{microtype}
  \UseMicrotypeSet[protrusion]{basicmath} % disable protrusion for tt fonts
}{}
\makeatletter
\@ifundefined{KOMAClassName}{% if non-KOMA class
  \IfFileExists{parskip.sty}{%
    \usepackage{parskip}
  }{% else
    \setlength{\parindent}{0pt}
    \setlength{\parskip}{6pt plus 2pt minus 1pt}}
}{% if KOMA class
  \KOMAoptions{parskip=half}}
\makeatother
\usepackage{xcolor}
\IfFileExists{xurl.sty}{\usepackage{xurl}}{} % add URL line breaks if available
\IfFileExists{bookmark.sty}{\usepackage{bookmark}}{\usepackage{hyperref}}
\hypersetup{
  pdftitle={relatorio},
  pdfauthor={Luísa},
  hidelinks,
  pdfcreator={LaTeX via pandoc}}
\urlstyle{same} % disable monospaced font for URLs
\usepackage[margin=1in]{geometry}
\usepackage{graphicx}
\makeatletter
\def\maxwidth{\ifdim\Gin@nat@width>\linewidth\linewidth\else\Gin@nat@width\fi}
\def\maxheight{\ifdim\Gin@nat@height>\textheight\textheight\else\Gin@nat@height\fi}
\makeatother
% Scale images if necessary, so that they will not overflow the page
% margins by default, and it is still possible to overwrite the defaults
% using explicit options in \includegraphics[width, height, ...]{}
\setkeys{Gin}{width=\maxwidth,height=\maxheight,keepaspectratio}
% Set default figure placement to htbp
\makeatletter
\def\fps@figure{htbp}
\makeatother
\setlength{\emergencystretch}{3em} % prevent overfull lines
\providecommand{\tightlist}{%
  \setlength{\itemsep}{0pt}\setlength{\parskip}{0pt}}
\setcounter{secnumdepth}{-\maxdimen} % remove section numbering
\ifluatex
  \usepackage{selnolig}  % disable illegal ligatures
\fi

\title{relatorio}
\author{Luísa}
\date{05/12/2021}

\begin{document}
\maketitle

\hypertarget{relatuxf3rio-final-diuxe1rio-do-clima}{%
\section{Relatório Final Diário do
Clima}\label{relatuxf3rio-final-diuxe1rio-do-clima}}

\textbf{Equipe OKBR:}

\begin{itemize}
\tightlist
\item
  Andre Assumpção;
\item
  Ariane Alves;
\item
  Fernanda Campagnucci;
\item
  Giulio Carvalho;
\item
  Juliana Trevine;
\item
  Luísa Coelho;
\item
  Murilo Machado;
\item
  Pedro Guimarães
\end{itemize}

\hypertarget{resumo}{%
\subsubsection{Resumo}\label{resumo}}

Proposta de relatório final do projeto Diário do Clima.

\hypertarget{introduuxe7uxe3o}{%
\subsection{Introdução}\label{introduuxe7uxe3o}}

O Diário Oficial da União (DOU) é um importante veículo de comunicação
nacional, criado em 1° de outubro de 1862, sancionado pela Lei Imperial
nº 1.177, após 54 anos da chegada da corte portuguesa ao Brasil. Sua
função, desde então, é divulgar as leis e normas do país, de forma
simples, direta e documental, sem qualquer tipo de contestação ou
queixa, sendo este o instrumento em que o governo presta contas à
sociedade (\href{link}{Lara, 20xx}). Atualmente o DOU pode ser acessado
através do \href{https://www.gov.br/imprensanacional/pt-br}{link da
Imprensa Nacional} permitindo buscas por palavras-chave ou até mesmo uma
busca avançada que permite o filtro por seções, data, busca por títulos
ou conteúdo, entre outros. As seções básicas do DOU consistem em:

\begin{enumerate}
\def\labelenumi{\arabic{enumi}.}
\tightlist
\item
  Leis, decretos, resoluções, instruções normativas, portarias e outros
  atos normativos de interesse geral;
\item
  Atos de interesse dos servidores da Administração Pública Federal;
\item
  Extratos de instrumentos contratuais (acordos, ajustes, autorizações
  de compra, contratos, convênios, ordens de execução de serviço, termos
  aditivos e instrumentos congêneres) editais de citação, intimação,
  notificação e concursos públicos, comunicados, avisos de licitação
  entre outros atos da administração pública decorrentes de disposição
  legal.
\end{enumerate}

Por conter tantas informações relevantes, o DOU é utilizado como fonte
em diferentes níveis de pesquisa, principalmente para recuperação de
informações.
\href{https://www.scielo.br/j/rap/a/cXjcx6hgZ4r5XN3t6rd7SSJ/?lang=en}{Guerra
et al.~(2015)}, por exemplo, analisa a resposta das agências reguladoras
no combate a COVID-19 através da busca de conteúdo no DOU e também das
11 agências reguladoras no país. Outros estudos contemplam a análise
regulatória de medicamentos no Brasil
\href{http://dx.doi.org/10.11606/issn.2316-9044.v18i2p122-156}{(Feitoza-Silva
et al., 2017)} e a inferência sobre risco de fraude nos contratos
públicos
\href{https://aclanthology.org/2020.findings-emnlp.143.pdf}{(Lima et
al., 2020)}

No entanto, poucos estudos mencionam Diários Oficiais Municipais.
\href{https://www.scielo.br/j/tinf/a/mJmTKbL94hj89q9p8HfCnLj/?format=pdf\&lang=pt}{Xavier
et al.~(2015)} propuseram uma arquitetura híbrida de indexação do Diário
Municipal de Cachoeiro de Itapemirim - ES,

No cenário da administração pública municipal, o setor do Diário Oficial
Municipal tem a atribuição de indexar e publicar todo o conteúdo gerado
pelo Poder Executivo através dos Atos Normativos contidos em uma edição
do Diário Oficial. Trata-se de uma tarefa nada trivial que consome tempo
e recursos do setor, e quase sempre é auxiliada por ferramentas
computacionais não específicas, o que compromete ainda mais a eficiência
do processo.

Para que um DO município serve? Explicação do projeto - GNI + Diário do
Clima; Sumário de resultados;

O projeto Diário do Clima selecionado a partir de um edital da
\href{https://newsinitiative.withgoogle.com/}{Google News Initiative}
tem como objetivo gerar uma plataforma de consulta sobre iniciativas e
políticas de combate à mudanças climáticas nos 5.570 municípios
brasileiros no qual culminará em uma importante fonte para jornalistas.
Para isso será necessário segmentar informações sobre políticas e
iniciativas de combate à mudanças climáticas nos diários oficiais dos
municípios brasileiros através de mecanismos de inteligência artificial.

\hypertarget{explorando-o-querido-diuxe1rio}{%
\subsubsection{Explorando o Querido
Diário}\label{explorando-o-querido-diuxe1rio}}

Por lei, todo ato público, seja do Poder Executivo, seja do Legislativo
ou do Judiciário, deve ser publicado em um diário oficial. Contudo, não
existe padronização de diários oficiais no Brasil nem uma central de
dados que permita que a sociedade civil acesse o conteúdo de diários de
maneira fácil e rápida. Para libertar os dados dos diários municipais, a
Open Knowledge Brasil (OKBR) criou o projeto ``Querido Diário'', cujo
objetivo é mapear, coletar e processar todos os diários municipais do
país e disponibilizá-los em formato aberto para a sociedade civil. Para
realizar essas três tarefas, a Open Knowledge Brasil conta com uma
equipe de pessoas cientistas e engenheiras de dados e uma grande
comunidade de colaboradores voluntários.

Criação do Querido Diário; Mapeamento dos diários Raspagem e
estruturação dos dados Utilização e análise do produto

Histórico (o que veio com a pandemia/Fundação Lemann/Lemann - Ensino
remoto/Diário do Clima) +
\href{https://queridodiario.ok.org.br/sobre\#historia}{Diferentes
projetos}: \ldots{} Como funciona

\hypertarget{diuxe1rio-do-clima}{%
\subsubsection{Diário do Clima}\label{diuxe1rio-do-clima}}

Problema a ser resolvido - mapear; - Como surgiu + GNI; - Responsáveis
pelo projeto - organizações parceiras; - Principais diferenças do
Querido Diário; - Quais informações serão coletadas? - Escopo/objetivo
do diário do clima.

\hypertarget{marcos-regulatuxf3rios---meio-ambiente}{%
\subsubsection{Marcos regulatórios - Meio
Ambiente}\label{marcos-regulatuxf3rios---meio-ambiente}}

\begin{itemize}
\item
  \href{https://www.planalto.gov.br/ccivil_03/Constituicao/Constituicao.htm}{Constituição
  Federal de 1988}

  \begin{itemize}
  \tightlist
  \item
    Art. 24 Inciso 3° Inexistindo lei federal sobre normas gerais, os
    Estados exercerão a competência legislativa plena, para atender a
    suas peculiaridades
  \item
    Art. 30. Compete aos Municípios: I - legislar sobre assuntos de
    interesse local; II -- suplementar a legislação federal e a estadual
    no que couber (\ldots)
  \item
    Arts. 34 e 35: Das intervençoes feitas pela União;
  \item
    {[}Lei de Crimes Ambientais
    (1998){]}{[}\url{http://www.planalto.gov.br/ccivil_03/LEIS/L9605.htm}{]}:
    Art. 73. Os valores arrecadados em pagamento de multas por infração
    ambiental serão revertidos ao Fundo Nacional do Meio Ambiente,
    criado pela
    \href{http://www.planalto.gov.br/ccivil_03/LEIS/L7797.htm}{Lei nº
    7.797, de 10 de julho de 1989}, Fundo Naval, criado pelo
    \href{https://legislacao.planalto.gov.br/legisla/legislacao.nsf/viwTodos/04737B762935FE7A032569FA0045A2E0?OpenDocument\&HIGHLIGHT=1,}{Decreto
    nº 20.923, de 8 de janeiro de 1932}, fundos estaduais ou municipais
    de meio ambiente, ou correlatos, conforme dispuser o órgão
    arrecadador.
  \end{itemize}
\item
  Política Nacional do Meio Ambiente
  (\href{https://www.planalto.gov.br/ccivil_03/LEIS/L6938.htm}{Lei
  6.938/81}): Visa desenvolvimento sócio-econômico atendendo aos
  princípios de ação governamental, racionalização, planejamento,
  proteção, controle e zoneamento das atividades, proteção e educação
  (Art. 2°) . Art 5º - As diretrizes da Política Nacional do Meio
  Ambiente serão formuladas em normas e planos, destinados a orientar a
  ação dos Governos da União, dos Estados, do Distrito Federal, dos
  Territórios e dos Municípios no que se relaciona com a preservação da
  qualidade ambiental e manutenção do equilíbrio ecológico, observados
  os princípios estabelecidos no art. 2º desta Lei.

  \begin{itemize}
  \item
    SISNAMA

    \begin{itemize}
    \tightlist
    \item
      Órgão Superior: Assessora Presidente da República na formulação de
      políticas (CSMA - Conselho Superior do Meio Ambiente)

      \begin{itemize}
      \tightlist
      \item
        Órgão consultivo e deliberativo: CONAMA - determina, quando
        julga necessário, a realização de estudos das alternativas e das
        possíveis consequências ambientais de projetos públicos ou
        privados, requisitando aos órgãos federais, estaduais e
        municipais, bem assim a entidades privadas, as informações
        indispensáveis para apreciação dos estudos de impacto ambiental,
        e respectivos relatórios, no caso de obras ou atividades de
        significativa degradação ambiental, especialmente nas áreas
        consideradas patrimônio nacional. -\textbf{PODER DE LEGISLAR}
      \item
        Órgão central: Secretaria do Meio Ambiente da Presidência da
        República - Ministério do Meio Ambiente;
      \item
        Órgãos executores: IBAMA e Instituto Chico Mendes
      \item
        Órgãos setoriais: órgãos ou entidades integrantes da
        Administração Pública Federal, direta ou indireta, bem como as
        fundações instituídas pelo Poder Público, cujas entidades
        estejam, total ou parcialmente, associadas às de preservação da
        qualidade ambiental ou de disciplinamento do uso de recursos
        ambientais
      \end{itemize}
    \item
      Órgãos Seccionais: órgãos ou entidades estaduais responsáveis pela
      execução de programas, projetos e pelo controle e fiscalização de
      atividades capazes de provocar a degradação ambiental.
      (Secretarias Estaduais de Meio Ambiente, IMA, IAP, CETESB, Inea,
      Polícia Militar Ambiental)
    \item
      Órgãos Locais: órgãos ou entidades municipais, responsáveis pelo
      controle e fiscalização dessas atividades, nas suas respectivas
      jurisdições
    \end{itemize}
  \item
    Fica alterado:

    Art. 3° da
    \href{https://www.planalto.gov.br/ccivil_03/LEIS/L10165.htm\#art1}{Lei
    n° 10.165, de 27 de dezembro de 2000}: A Lei n° 6.938, de 1981,
    passa a vigorar acrescida dos seguintes artigos:
    \href{https://www.planalto.gov.br/ccivil_03/LEIS/L6938.htm\#art17q}{"Art.
    17-Q}. É o Ibama autorizado a celebrar convênios com os Estados, os
    Municípios e o Distrito Federal para desempenharem atividades de
    fiscalização ambiental, podendo repassar-lhes parcela da receita
    obtida com a TCFA." \textbf{Transferência de \emph{recurso com
    pessoal}}

    \begin{quote}
    obrigatoriedade de que todos os as propriedades e posses rurais do
    País façam parte do Sistema Nacional de Cadastro Ambiental Rural
    (Sicar)
    \end{quote}

    Art. 3o da
    \href{https://www.planalto.gov.br/ccivil_03/LEIS/L10165.htm\#art1}{Lei
    n° 10.165, de 27 de dezembro de 2000}: A Lei n° 6.938, de 1981 passa
    a vigorar acrescida dos seguintes
    \href{https://www.planalto.gov.br/ccivil_03/LEIS/L10165.htm\#anexoviii}{Anexos
    VIII} (atividades potencialmente poluidoras) e
    \href{https://www.planalto.gov.br/ccivil_03/LEIS/L10165.htm\#anexoix}{IX:}(TCFA)
  \end{itemize}

  \hypertarget{uxf3rguxe3os}{%
  \paragraph{Órgãos}\label{uxf3rguxe3os}}

  \emph{MMA}: criado em 1992 (marco da Rio 92)

  \begin{itemize}
  \item
    Lei das águas (1997):
  \item
    Lei de Crimes Ambientais (1998):
  \item
    Política Nacional de Educação Ambiental (1999):
  \item
    Sistema Nacional de Unidades de Conservação (SNUC)(2000):
  \item
    Lei de Gestão de Florestas Públicas (2006):
  \end{itemize}

  \emph{IBAMA}: Poder de polícia ambiental, responsável por recolher e
  repassar recursos da TCFA, ITR, CAR, APP, RL. objetivos institucionais
  relativos ao licenciamento ambiental, ao controle da qualidade
  ambiental, à autorização de uso dos recursos naturais e à
  fiscalização, monitoramento e controle ambiental e ações supletivas de
  competência da União, conforme legislação ambiental. Zoneamento e
  avaliação de impacto ambiental. CTF

  \begin{itemize}
  \tightlist
  \item
    APP: Área de preservação permanente
  \item
    AUR: Área de Uso Restrito
  \item
    RL: Reserva Legal
  \item
    CAR: Cadastro Ambiental Rural é o registro público eletrônico de
    âmbito nacional, obrigatório para todos os imóveis rurais, com a
    finalidade de integrar as informações ambientais das propriedades e
    posses rurais referentes às APP, AUR, RL, de remanescentes de
    florestas e demais formas de vegetação nativa, e das áreas
    consolidadas, compondo base de dados para controle, monitoramento,
    planejamento ambiental e econômico e combate ao desmatamento.

    \begin{itemize}
    \tightlist
    \item
      PRA: Programa de Apoio e Incentivo à Conservação do Meio Ambiente
      e aos Programas de Regularização Ambiental
    \item
      ITR: O CAR gera créditos tributários por meio de deduções na base
      de cálculo do Imposto Territorial Rural
    \end{itemize}
  \item
    CTF: Cadastro Técnico Federal identifica as pessoas físicas e
    jurídicas sob controle ambiental e fiscalização ambiental, conforme
    previsto em legislação federal ou de âmbito nacional, gerando
    informações para a gestão ambiental no Brasil.

    \begin{itemize}
    \tightlist
    \item
      TCFA: Taxa de Controle e Fiscalização Ambiental é um tributo para
      controle e fiscalização das atividades potencialmente poluidoras e
      utilizadoras de recursos naturais
    \end{itemize}
  \end{itemize}

  \emph{ICMBio}: apresentar e editar normas e padrões de gestão de
  Unidades de Conservação (UC) federais; propor a criação, regularização
  fundiária e gestão das UC federais; e apoiar a implementação do
  Sistema Nacional de Unidades de Conservação (SNUC). Contribui para a
  recuperação de áreas degradadas em UC. Fiscaliza e aplica penalidades
  administrativas ambientais ou compensatórias aos responsáveis pelo não
  cumprimento das medidas necessárias à preservação ou correção da
  degradação ambiental. Monitora o uso público e a exploração econômica
  dos recursos naturais nas UC onde isso for permitido, obedecidas as
  exigências legais e de sustentabilidade do meio ambiente. geração e
  disseminação sistemática de informações e conhecimentos relativos à
  gestão de UCs, da conservação da biodiversidade e do uso dos recursos
  faunísticos, pesqueiros e florestais. Contribui para a implementação
  do Sistema Nacional de Informações sobre o Meio Ambiente (Sinima) e
  aplica, no âmbito de sua competência, dispositivos e acordos
  internacionais relativos à gestão ambiental. Elabora o Relatório de
  Gestão das UC.

  \begin{itemize}
  \item
    UC: Unidade de Conservação

    \emph{Setor privado}
  \item
    Principal: ISO da série 14000;
  \item
    Outros: Certificação Orgânica; Comércio Justo; EurepGap (Boas
    Práticas Agrícolas (GAP); Certificação Socioambiental e Produção
    Integrada
  \end{itemize}

  \hypertarget{instrumentos-da-pnma}{%
  \paragraph{Instrumentos da PNMA:}\label{instrumentos-da-pnma}}

  \begin{itemize}
  \item
    Padrões de qualidade ambiental;
  \item
    Zoneamento ambiental;
  \item
    Avaliação de impactos ambientais;
  \item
    Licenciamento e revisão de atividades efetiva ou potencialmente
    poluidoras;
  \item
    Incentivos à produção e instalação de equipamentos e a criação ou
    absorção de tecnologia, voltados para a melhoria da qualidade
    ambiental;
  \item
    criação de espaços territoriais especialmente protegidos pelo Poder
    Público federal;
  \item
    Sistema nacional de informações sobre o meio ambiente;
  \item
    Cadastro Técnico Federal de Atividades e Instrumentos de Defesa
    Ambiental;
  \item
    penalidades disciplinares ou compensatórias ao não cumprimento das
    medidas necessárias à preservação ou correção da degradação
    ambiental;
  \item
    instituição do Relatório de Qualidade do Meio Ambiente, a ser
    divulgado anualmente pelo IBAMA;
  \item
    garantia da prestação de informações relativas ao Meio Ambiente,
    obrigando-se o Poder Público a produzí-las, quando inexistentes.
  \item
    Cadastro Técnico Federal de atividades potencialmente poluidoras
    e/ou utilizadoras dos recursos ambientais.
  \item
    instrumentos econômicos, como concessão florestal, servidão
    ambiental, seguro ambiental e outros.
  \end{itemize}
\end{itemize}

\hypertarget{planos-e-programas-do-mma}{%
\subsubsection{Planos e Programas do
MMA}\label{planos-e-programas-do-mma}}

\begin{itemize}
\tightlist
\item
  {[}Programa
  Floresta+{]}{[}\url{https://www.gov.br/mma/pt-br/assuntos/servicosambientais/florestamais/ProgramaFloresta.pdf}{]};
\item
  {[}Plano Nacional para Controle do Desmatamento Ilegal e Recuperação
  da Vegetação
  Nativa{]}{[}\url{https://www.gov.br/mma/pt-br/assuntos/servicosambientais/controle-de-desmatamento-e-incendios-florestais/PlanoNacionalparaControledoDesmatamento20202023.pdf}{]}
  existem diretrizes para os municípios, consequentemente no {[}Plano
  Operativo 2020 -
  2023{]}{[}\url{https://www.gov.br/mma/pt-br/assuntos/servicosambientais/controle-de-desmatamento-e-incendios-florestais/PlanoOperativo20202023.pdf}{]}
  - Implementar programas e projetos de Pagamentos por Serviços
  Ambientais (PSA) existe responsabilidade das prefeituras municipais;

  \begin{itemize}
  \tightlist
  \item
    Nenhum dos municípios descritos
    {[}aqui{]}{[}\url{https://www.gov.br/mma/pt-br/assuntos/servicosambientais/controle-de-desmatamento-e-incendios-florestais/pdf/Listagemmunicpiosprioritriosparaaesdepreveno2021.pdf}{]}
    estão na amostra inicial.
  \end{itemize}
\item
  {[}PLANAVEG Plano Nacional de Recuperação da Vegetação
  Nativa{]}{[}\url{https://www.gov.br/mma/pt-br/assuntos/servicosambientais/ecossistemas-1/planaveg_plano_nacional_recuperacao_vegetacao_nativa.pdf}{]}
  - Entidades municipais como Secretarias de Meio Ambiente e Conselhos
  Municipais de Meio Ambiente para atuar na integração entre estâncias,
  sensibilização da população, e capacitação, bem como recuperação de
  áreas.
\item
  {[}REDD++{]}{[}\url{http://redd.mma.gov.br/images/central-de-midia/pdf/publicacoes/notainformativa2018_captacaodescentralizacao.pdf}{]};
\item
  {[}ARPA, A3P, Cerrado Sustentável, Zoneamento ecológico, dentre
  outros{]}{[}\url{https://antigo.mma.gov.br/programas-mma.html}{]};
\end{itemize}

\hypertarget{metodologia-da-pesquisa}{%
\subsection{Metodologia da Pesquisa}\label{metodologia-da-pesquisa}}

A recuperação de informações (RI) pode ser feita através de um sistema
que funciona a partir de diferente mecanismos, sejam análise textual,
filtragem de informação por meio da extração de stopwords, técnicas de
redução de palavras a seus radicais (stemming), técnicas de indexação,
arquivo invertido, modelos matemáticos e estatísticos para a
representação de documentos, de consultas e a definição de coeficientes
de similaridades, estruturas de categorização e de expansão de consultas
por meio da utilização de Thesaurus (Aires, 2005).

Etapas: Pré-processamento, tokenização (curadoria), stopwords e
stoplists, normalização, pós-processamento: avaliação de precisão

\[
Precisão = n° de documentos relevantes/n° de documentos recuperados
\]

\ldots{} \ldots{}

\end{document}
